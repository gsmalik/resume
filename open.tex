\begin{enumerate}
    % hopliteml
    \item Hoplite ML: \href{https://git.uwaterloo.ca/watcag-public/hoplite-ml}{https://git.uwaterloo.ca/watcag-public/hoplite-ml}
    
    Learning the design of Hoplite overlays using Maximum Likelihood Estimation. This is the open sourced repository of the work in the paper "Learn the Switches: Evolving FPGA NoCs with Stall-Free and Backpressure Based Routers".
    
    % ctci-python
    \item CTCI Python: \href{https://github.com/gsmalik/ctci_python}{https://github.com/gsmalik/ctci\_python}
    
    This is fun repository that implements solutions to almost all programming questions in the infamous "Cracking the coding interview". All solutions are implemented in python, with detailed asymptotic analysis of time and space complexities of the solutions.
    
    % keras-lmu
    \item Keras LMU: \href{https://github.com/nengo/keras-lmu}{https://github.com/nengo/keras-lmu}
    
    Keras-based implementation of the Legendre Memory Unit (LMU). The LMU is a novel memory cell for recurrent neural networks that dynamically maintains information across long windows of time using relatively few resources.
    
    % darwinn
    \item DarwiNN: \href{https://github.com/Xilinx/DarwiNN}{https://github.com/Xilinx/DarwiNN}
    
    DarwiNN is a toolbox of functions enabling the training of DNN models using Evolutionary Strategies (ES). DarwiNN is built on top of PyTorch, enabling easy integration into DNN training flows. DarwiNN provides several GPU-accelerated evolutionary primitives - mutation, recombination - that enable fast execution of multiple flavors of ES, as well as support for distributed execution of neuroevolution.
    
    % partitioning NN
    \item Neural Network Mapping: \href{https://git.uwaterloo.ca/watcag-public/fpga-syspart}{https://git.uwaterloo.ca/watcag-public/fpga-syspart}
    
    Learning the mapping of a given neural network to a systolic array using Co-variance Matrix Adaptation Evolutionary Strategies (CMA-ES). This is the open sourced repository of the work in the paper "Partitioning Systolic Arrays for Fun and Profit".
    
    % bft flow
    \item BFT Flow: \href{https://git.uwaterloo.ca/watcag-public/bft-flow}{https://git.uwaterloo.ca/watcag-public/bft-flow}
    
    Implementing butterfly fat trees overlays. This is the open sourced repository of the work in the paper "Enhancing butterfly fat tree NoCs for FPGAs with lightweight flow control".
\end{enumerate}

